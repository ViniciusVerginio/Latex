\documentclass{article}
\usepackage[utf8]{inputenc}
\usepackage{amsmath}
\usepackage{indentfirst}
\author{Vinícius Verginio}
\date{15/04/2013}
\title{SHOW LITTLE BOY}

\newcommand{\vetor}{\textbf{E}} %troca texbr para vetor

\begin{document}

\maketitle

\newpage
\section{Introdução}
\label{sec:intro} %cria rotulo para seção
Ainda não sei sobre o que será meu TCC. Estou sob pressão muito grande!!! %daqui pra frente nao vai no texto.

\section{Motivação}
\label{sec:motiv}
Conseguir o título de Eng. Eletricista.

Como vimos na introdução (secao \ref{sec:intro}) %da só o numero

\ref{sec:motiv}

Na equação $V=R I$

Outra equação $I=V/R$

Um numero em cima do outro fração => $I=\frac{V}{R}$

$$I=\frac{V}{R}$$
$ $%agora ta no modo matemático

$$0=1+e^{j\pi}$$ %tem que usar os colchetes para ficar em cima

IDFT {fórmula número \ref{eq:idft}: 
\begin{equation}
\label{eq:idft}
x_n=\frac{1}{N}\sum_{k=0}^{n-1}X_k\cdot e^{i2\pi kn/N}
\end{equation}

\begin{align}
\nabla \cdot E = - \frac{\partial \textbf{B}}{\partial t}
\mu_0 \textbf{J} + \mu_0 \epsilon_0 \frac{\partial E}{\partial t}
\end{align}
show

\end{document}
