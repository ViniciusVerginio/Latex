\documentclass{report}
\usepackage[utf8]{inputenc}
%outro tipo de usar é o "ansinew"
\usepackage{amsmath}
\usepackage{indentfirst}
\usepackage{multirow}
\usepackage{float}
\usepackage{graphicx}
\usepackage[brazil]{babel}
\usepackage{enumitem}
\usepackage{circuitikz}


\author{Vinícius Verginio}
\date{15/04/2013}
\title{SHOW LITTLE BOY}

\newcommand{\vetor}{\textbf{E}} %troca texbr para vetor

\begin{document}

\maketitle

\newpage
\chapter{Introdução}
\section{Introdução}
\label{sec:intro} %cria rotulo para seção
Ainda não sei sobre o que será meu TCC. Estou sob pressão muito grande!!! %daqui pra frente nao vai no texto.

acrescentar nova linha
\hfill estou bem sim %cria linha nova
\footnote{este arquivo mostra como funciona o latex} %rodapé

\begin{table}
\begin{tabular}{|l|l|l|r|} %a segunda parte é como vai ficar centrado
\multirow{2}{*}{ae}
	a & b & \multicolumn{2}{c}{cd} \\
	\cline{2-4}
	a & b & c & d \\ 
	
\end{tabular}
\end{table}

%cosmetico - usar commit com este nome para arrumar espaços, carateres e outros. Coisas que nao mudem o resultado final.

%figura
%\graphicspath{{Desktop/}}
%\begin{figure}[!htb]
%\centering
%\includegraphics{Ju 120 kg}
%\caption{antigamente}
%\label{fig:jucelia}
%\end{figure}

%Capitulos

%enumerar
\begin{enumerate}
	\item avai
	\item criciuma
		\begin{enumerate}[label*=\arabic*.]
		\item angeloni
		\item valter		
		\end{enumerate}
\end{enumerate}

%desenho de circuito elétrico -> ver bibliotec circuitikz
\chapter{Circuitos}
\label{fig:circuito}
O circuito esta apresentado na figura \ref{fig:circuito} %falta fazer referencia

\begin{figure}[!htb]
\centering
\begin{tikzpicture}
\draw (0,0) to[lamp] (2,0); %do ponto 0,0 ate 0 2,0 coloque uma lampada
\end{tikzpicture}
\end{figure}

outro circuito
\begin{figure}[!htb]
\centering
\begin{tikzpicture}
\draw (0,0) to[sV] (0,2) to[closing switch] (2,2) to[lamp] (2,0) -- (0,0);
\end{tikzpicture}
\end{figure}


\newpage
\chapter{Motivação}
\section{Motivação}
\label{sec:motiv}
Conseguir o título de Eng. Eletricista.
Como vimos na introdução (secao \ref{sec:intro}) %da só o numero

\ref{sec:motiv}

Na equação $V=R I$
Outra equação $I=V/R$
Um numero em cima do outro fração => $I=\frac{V}{R}$

$$I=\frac{V}{R}$$
$ $%agora ta no modo matemático

$$0=1+e^{j\pi}$$ %tem que usar os colchetes para ficar em cima

IDFT {fórmula número \ref{eq:idft}: 
\begin{equation}
\label{eq:idft}
x_n=\frac{1}{N}\sum_{k=0}^{n-1}X_k\cdot e^{i2\pi kn/N}
\end{equation}
%aling8 sem o numero do lado da equação
equações de \ref{eq:max1} a \ref{eq:max2}

\begin{align}
\label{eq:max1}
\nabla \cdot E = - \frac{\partial \textbf{B}}{\partial t}\\
\label{eq:max2}
\mu_0 \textbf{J} + \mu_0 \epsilon_0 \frac{\partial E}{\partial t}
\end{align}

\end{document}
